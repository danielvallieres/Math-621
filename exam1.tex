\documentclass[reqno]{amsart} 
\usepackage{amssymb,latexsym,amsmath,amscd,graphicx,setspace,amsthm,verbatim}
\usepackage[margin = 3 cm]{geometry}


\theoremstyle{plain}
\newtheorem{theorem}{Theorem}[section]
\newtheorem{proposition}{Proposition}
\newtheorem{corollary}{Corollary}
\newtheorem{lemma}{Lemma}
\newtheorem{conjecture}{Conjecture}
\newtheorem{question}{Question}
\newtheorem{problem}{Problem}
      
\theoremstyle{definition}
\newtheorem{definition}{Definition}

\newenvironment{solution}{\paragraph{\emph{Solution}.}}{\hfill$\square$}


\newenvironment{solution1}{\paragraph{\emph{Solution $1$}.}}{\hfill$\square$}
\newenvironment{solution2}{\paragraph{\emph{Solution $2$}.}}{\hfill$\square$}
\newenvironment{solution3}{\paragraph{\emph{Solution $3$}.}}{\hfill$\square$}

\begin{document} 

\title[Exam 1]{Exam 1}

\date{\today} 
\maketitle 


\begin{problem}
Suppose $\mathcal{S}$ is the smallest $\sigma$-algebra on $\mathbb{R}$ containing
$$\{(r,n]: r \in \mathbb{Q}, n \in \mathbb{Z} \}. $$
Prove that $\mathcal{S}$ is the collection of Borel subsets of $\mathbb{R}$.
\end{problem}
\begin{solution}

\end{solution}


\begin{problem}
Give an example of a measurable space $(X,\mathcal{S})$ and a function $f:X \rightarrow \mathbb{R}$ such that $|f|$ is $\mathcal{S}$-measurable but $f$ is not $\mathcal{S}$-measurable.
\end{problem}
\begin{solution}

\end{solution}

\begin{problem}
Let $A \subseteq \mathbb{R}$.  Show that $A \in \mathcal{L}(\mathbb{R})$ if and only if for all $\varepsilon > 0$, there exist an open set $G \subseteq \mathbb{R}$ and a closed set $F \subseteq \mathbb{R}$ such that $F \subseteq A \subseteq G$ and $\lambda(G \smallsetminus F) < \varepsilon$.
\end{problem}
\begin{solution}

\end{solution}


\begin{problem}
Construct a new Cantor set, say $C_{4}$, by starting with $[0,1]$ and removing open middle intervals of relative length $1/4$ at each stage.
\begin{enumerate}
\item Draw $G_{1}, G_{2}, G_{3}$ as we did in class for the usual Cantor set.
\item What is $\lambda(C_{4})$?  Explain carefully.
\end{enumerate}

\end{problem}
\begin{solution}

\end{solution}

\begin{problem}
Suppose $(X,\mathcal{S},\mu)$ is a measure space such that $\mu(X) < \infty$.  Prove that if $\mathcal{A}$ is a set of disjoint sets in $\mathcal{S}$ such that $\mu(A)> 0$ for every $A \in \mathcal{A}$, then $\mathcal{A}$ is a countable set.
\end{problem}
\begin{solution}

\end{solution}

\begin{problem}
Let $\mathcal{G}_{0}$ be the family of all open subsets of $\mathbb{R}$, and $\mathcal{F}_{0}$ be the family of all closed subsets of $\mathbb{R}$.  For $n \ge 1$, define
$$\mathcal{F}_{n} = \left\{\bigcup_{k=1}^{\infty}A_{k}: A_{k} \in \mathcal{F}_{n-1} \right\} \text{ and } \mathcal{G}_{n} = \left\{\bigcap_{k=1}^{\infty}A_{k}: A_{k} \in \mathcal{G}_{n-1} \right\}.$$
A set in $\mathcal{F}_{n}$ is called an $F_{\sigma_{n}}$-set and one in $\mathcal{G}_{n}$ is called a $G_{\delta_{n}}$-set.
\begin{enumerate}
\item Prove that $\mathcal{F}_{n} \cup \mathcal{G}_{n} \subseteq \mathcal{F}_{n+1} \cap \mathcal{G}_{n+1}$ for all $n \ge 0$.
\item Show that
$$\mathcal{B}(\mathbb{R}) = \bigcup_{n=0}^{\infty}(\mathcal{F}_{n} \cup \mathcal{G}_{n}). $$
\end{enumerate}

\end{problem}
\begin{solution}

\end{solution}



\begin{problem}
Show that if $A \subseteq \mathbb{R}$, then $A \in \mathcal{L}(\mathbb{R})$ if and only if 
$$\lambda^{*}(E) = \lambda^{*}(E \cap A) + \lambda^{*}(E \cap A^{c}) $$
for all $E \subseteq \mathbb{R}$.
\end{problem}
\begin{solution}

\end{solution}















\end{document} 



